\documentclass[a4paper, draft, reqno, 12pt, openbib]{article} % document class amsart class, draft points out the lines that are two long, reqno puts equations numbers to the right
\usepackage{geometry}\geometry{a4paper,total={170mm,250mm},left=20mm,top=25mm} % define margins of the document
\usepackage{graphicx} % package to load illustrations
\usepackage{listings} % package to read a programming language if included
\usepackage[utf8]{inputenc}
\usepackage[english]{babel}
\usepackage{color} 
\usepackage{hyperref} % enable hyperlinks in the document 
\hypersetup{
    colorlinks = true,
    linktoc = all,
    citecolor=green,
    filecolor=black,
    linkcolor=blue,
    urlcolor=red
}
% --------------------------- PREAMBLE --------------------------------------------%
\begin{document}
\title{Physics for mechanical engineers notes}
\author{Pradeep Janakiraman \thanks{TU Delft student from 2019-2021, notes written in Quarter 2 of AY 2020-2021}\\
3ME department\\
TU Delft, The Netherlands\\
pradeepjanakiraman@student.tudelft.nl}
\date{7 November 2020}
\maketitle
%---------------------------- TITLE PAGE ------------------------------------------%
\newpage
\tableofcontents
\newpage
%---------------------------- TABLE OF CONTENTS ------------------------------------%
\section{Introduction}
This document is a compilation of the notes that were taken while I prepared for the ME46006, Physics for Mechanical Engineers course exam. It was written with the intention to revise as I was copying over some notes in the hopes that I would review some of the material while typing this out. The organisation of the document is simple, it starts with the necessary review of necessary mathematical prerequisites. These are mainly from the excellent resource that is \href{https://www.feynmanlectures.caltech.edu/II_toc.html}{Feynman's lectures on Physics, Volume 2}. In fact, much of the course material is covered in these volume and it can be used as an additional reference, as I have done. The remaining part of this document covers the lectures that were covered in order and my summaries for each of them, example problems and homework problems that were assigned.
\section{  
	
	
	
	
	
	
	
\end{document}