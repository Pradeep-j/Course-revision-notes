\section{Lecture 1 - Electrostatics}
We first define an electric field $\vec{E}$ as the electrical force on a unit charge at position $2$ due to a to a point charge $q$ at position $1$. From coloumb's law, we can then write: 
\begin{equation}
	\label{formula for electric field}
	\vec{E} = \frac{1}{4\pi \epsilon_{0}} \frac{q}{r^{2}} \vec{e}_{12}
\end{equation}
$\vec{e}_{12}$ is the unit vector in the direction of the straight line between position 1 and 2. It should be stressed that $E$ is a vector and as 3 components along the 3 coordinate axes. Placing a unit charge anywhere in the electric field would result in a force on the unit charge given by \autoref{formula for electric field}. We then consider the work done on a unit charge to move it from an initial position to another postion. We know that, 
\begin{equation}
	W = -\int_{b}^{a} \vec{F} \cdot d\vec{s}
\end{equation}
Since we are only considering unit charges however, this can be re-written as: 
\begin{equation}
	\label{work equation}
	W = -\int_{b}^{a} \vec{E} \cdot d\vec{s}
\end{equation}
The above equation is really 3 integral formulas, one for each component. Now we note that the vector field $\vec{E}$ has some properties. 
\begin{itemize}
	\item The vector field is symmetric about the source point charge. 
	\item The vector field points in the radial direction with the source point charge at the center
\end{itemize}
So although the integal in \autoref{work equation} leads to different results depending on the path from $a$ to $b$, because of these two properties the integral becomes independent of the path and only depends on the positions of $a$ and $b$. Because of this we can write the result of the integral as follows: 
\begin{equation}
	-\int_{b}^{a} \vec{E} \cdot d\vec{s} = \phi(b) - \phi(a)
\end{equation}
$\phi(a, b)$ is the electric potential at points $a$ and $b$. From the figure below, we can see why this would be true. If we force the path we take from $a$ to $b$ to pass through $P_{0}$ we would get the result above. Usually $P_{0}$ is taken at $\infty$. 
\begin{figure}[H]
	\centering 
	\includegraphics[scale = 0.5]{Lecture 1/images/path independence}
	\caption{A path moving from $a$ to $b$ is made to pass through $P_{0}$, which is usually taken at $\infty$ with 0 potential}
	\label{Illustration of the result obtained earlier}
\end{figure}
Then we can write: 
\begin{align}
	\phi(a) &= -\int_{\infty}^{a} \vec{E} \cdot d\vec{s}\\
	&= \frac{1}{4\pi \epsilon_{0}} \frac{q}{r_{a}} 
\end{align}
It should be noted that the above holds true only when the source point charge is at the origin. If the source is not at the origin, then the new distance must be taken instead of $r_{a}$. The principle of superpostion cen then be invoked for the case when there is more than once source point charge and we can write: 
\begin{align}
	\phi(1) &= \sum_{i} \frac{1}{4\pi \epsilon_{0}}\frac{q_{i}}{r_{1i}}\\
	&= \frac{1}{4\pi \epsilon_{0}}\int_{V(2)}\frac{\rho(2) dV_{2}}{r_{12}}
\end{align}
The second equation above uses the concept of a charge density $\rho$ that characterises many charges over a certain, summing up the charge density multiplied with elemental volumes would result in the total charge at the $r_{12}$. We can now go back to the result where the integral in \autoref{work equation} is path independent. We have already arrived at such a result previously in \autoref{line integral def}. It implies the following: 
\begin{align}
	-\int_{b}^{a} \vec{E} \cdot d\vec{s} &= \phi(b) - \phi(a)\\
	\int_{b}^{a} -\vec{\nabla}\phi \cdot d\vec{s} &= \phi(b) - \phi(a)\\
	\vec{E} &= -\vec{\nabla}\phi
\end{align}
However, we have already gone through the properties of double derivatives in \autoref{curl grad} and using that relationship, we can define: 
\begin{equation}
	\label{Maxwell eqn 2 for statics}
	\vec{\nabla} \times \vec{E} = 0
\end{equation}
Which is one Maxwell's equations for electrostatics. 
	
















